\section{Introduction}
%% Domain introduction, problem, contributions, and organization of the paper.

Internet of Things (IoT) is a domain that represents the next most exciting technological revolution since the Internet. IoT will bring endless opportunities and impact every corner of our planet. In the healthcare domain, IoT promises to bring personalized health tracking and monitoring ever closer to the consumers. This phenomena is evidenced in a recent Wall Street Journal (June, 29, 2015) article entitled "Staying Connected is Crucial to Staying Healthy". Modern smartphones and related devices now contain more sensors than ever before. Data from sensors can be collected more easily and more accurately. In 2014, it is estimated that 46 million people are using IoT-based health and fitness applications. Currently, the predominant IoT-based health applications are in sports and fitness. However, disease management or preventive care health applications are becoming more prevalent. For example, the real-time preventive care applications such as those for detecting fall in elderly patients is one of the active research areas due to the aging population \cite{Tacconi:2011}. This paper describes a new category of real-time preventive health application that aims to reduce risky behaviors due to intoxication by predicting blood alcohol content (BAC) non-invasively and in real-time.

The Centers for Disease Control and Prevention (CDC) reported that between the years 2008 and 2010, roughly two-thirds of adults were drinkers, but adults between the ages 18 and 24 were most commonly associated with consumption of more than 5 drinks in one day \cite{Schoenborn:2013}. There is also a known causal relationship between alcohol consumption and risky behaviors \cite{Assaad:2006}. One of the most dangerous, worldwide (citation needed) problems of drinking is drunk driving. This problem is not only a hazard to the drunk drivers, but to pedestrians and other drivers. It is reported by the Bureau of Transportation Statistics that in 2010, 47.2\% of pedestrian fatalities and 39.9\% of vehicle occupant fatalities are caused by drunk driving \cite{Chambers:2012}.

Adults often find themselves on outings where they plan on drinking, but also on later driving home. At dangerous levels of intoxication, it can be difficult to judge ones own level of intoxication. Instead it would be better to get a definitive measurement; of the BAC, or simply a binary response, "drunk" or "not drunk." To do this, one can buy a breathalyzer and blow into it on every outing where drinking is involved. This is not a socially beneficial choice. The other option is to use a simple smartphone application to manually calculate BAC. However these demand a great deal of involvement from the user. To be practical, it would be useful to have some sort of non-invasive and accurate monitoring system that will warn its user if they become too intoxicated. A study conducted in Washington University shows that electronic intervention programs are more successful at reducing college student drinking than a general alcohol awareness program \cite{Ward:2015}. This system can also be used to warn friends and family, or prevent the operation of the user's car.

The main contributions of the paper are:

\begin{itemize}
	\item A machine learning model for the prediction of intoxication level from smartwatch sensor data.
	\item A general Android-based gateway system that can collect data from any type of physical or virtual sensor accessible by the host smartphone.
	\item An overall, sensor middleware system, including a geographic visualization service, that may be useful for answering many other important research questions.
\end{itemize}

%****Summary****