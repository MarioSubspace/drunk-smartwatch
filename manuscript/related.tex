\section{Related Work}
%
%Include the current IoT middleware systems that you have surveyed. Should start this early.
%\input{related_work}

There is much research being done in the prediction or calculation of blood alcohol content. 

The most common way to calculate BAC is through the Widmark equation. Posey and Mozayani in their paper \cite{Posey:2007} discuss the Widmark equation and improvements on its use by observing 5 previously published models and one new model for BAC prediction. The Widmark equation predicts BAC using elimination rates and the Widmark factor. Current use of the Widmark equation and BAC prediction is discouraged by experts due to inaccuracies, usually overestimation. Mathematical calculation of BAC is still sometimes used in DUI court cases, as there are not many alternatives, but it requires lab test results, information about what the person was drinking, how much and how long, BMI, and other hard to find details such as elimination rates. Some of this information has to be estimated or generalized which causes the calculation to be inaccurate. Also, due to the amount of information required, such a method is unusable by the general public for personal use. While BAC prediction is generally frowned upon in court cases, Posey and Mozayani make a point that using several different models provides a more accurate prediction than using the original Widmark equation alone.

Another known method of measuring BAC is through a transdermal ethanol sensor. Webster and Gabler \cite{Webster:2007} discussed the feasibility of transdermal ethanol sensors being integrated into ignition interlock systems in order to prevent drunk driving. A way to reduce the number of incidents related to drunk driving is to somehow stop people from driving drunk. Their approach to do this is to fit every vehicle with a ignition interlock system that responds to the driver's BAC. Current interlock systems like this involve a breathalyzer, but these are seen as expensive, bulky, and elicit a negative response from the general public. Transdermal ethanol sensing provides a more discreet way to measure BAC. However, according to the paper, an issue with transdermal ethanol sensing is that there is a significant delay, up to roughly an hour, from the time of alcohol consumption and detection at the skin. 

In 2014, James Baldwin applied for a patent on a wearable blood alcohol measuring device \cite{Baldwin:2014} as a non-invasive, discreet, and wearable method of continuously predicting blood alcohol content. Baldwin's method involves the use of a transdermal ethanol sensor and a method of predicting future BAC given the user's intended consumption. It is also meant to provide guidance for the user as to when to stop in order to be at a certain BAC by a previously specified time. Baldwin's invention is very similar to this project with the only major difference being that it uses a wearable transdermal ethanol sensor to directly measure BAC instead of using common sensors found in smartwatches and a prediction model, such as we do, in order to predict BAC. The use of a transdermal ethanol sensor is proven to have many issues, mostly in the delay between consumption and sensing \cite{Webster:2007}. Another issue is that it requires a special sensor that would only be used for this purpose. Our project improves on this issue by using a common device that many already own for various other reasons. Also, due to our approach for monitoring drunkenness, our method may provide more immediate predictions.  

In 2015, Koukiou and Anastassopoulos published a paper on using a neural network to identify drunkenness from thermal infrared imagery \cite{Koukiou:2015}. Different neural networks were trained on different parts of the face in order to determine which areas can be used to discriminated "a drunk from a sober person." Their findings showed that the forehead was the main feature to change thermal behavior from the consumption of alcohol. Later, the whole face was used to train a neural structure which showed high performance on discrimination. The results showed that the forehead and the nose are the areas of the face that were most prominent in drunk people. A neural networks was then trained on the forehead of only one person which was then used on other persons in order to determine those who were drunk. The results had a success rate of around 90\%. This study shows that there is a correlation between skin temperature and drunkenness, which is applicable to our research because of the skin temperature sensor found on many smartwatches. This project has many similar applications as our would, such as being able to be incorporated into ignition interlock systems or to give evidence to authorities about intoxication in investigations. However, this system may not be able to be used by people to determine their own drunkenness as it requires a thermal infrared camera to be focused on the users face in order to do so. This would be difficult to do discreetly or conveniently. 

There is also research going into using various sensors for health applications, many using a machine learning approach to do so. 

In "Multi-Sensor Fusion for Enhanced Contextual Awareness of Everyday Activites with Ubiquitous Devices" \cite{Guiry:2014}, the authors attempt to use the sensors in smartphones and smartwatches to identify various daily activities such as walking, running, cycling, sitting, etc. It also identified whether or not the user is indoors or outdoors. Collected data was then used to train 5 different machine learning algorithms: C4.5, CART, Na\"{\i}ve Bayes, Multi-Layer Perceptrons, and Support Vector Machines. Results showed promise, with one model for inside/outside classification being 100\% correct. This study shows that the use of multiple sensors improves classification accuracy. For our research, we may be able to improve BAC prediction by recognizing when the user is taking a drink. This research shows that machine learning models can be used to label collected sensor data to certain activities.

Study has been done in using accelerometers to detect seizures \cite{Milosevic:2014} and using smartphones and smartwatches to monitor Parkinson Disease \cite{Sharma:2014}. In research on the detection of seizures \cite{Milosevic:2014}, researches use accelerometers on each wrist and ankle and a machine learning model in order to detect epileptic convulsions. Results included detection of 19 out of 23 seizures that were longer than 30 seconds each, with only a false alarm rate of .39 per hour. SPARK, a proposed framework that uses smartphones and smartwatches to monitor for symptoms of Parkinson Disease \cite{Sharma:2014}, including speech analysis, facial tremors, dyskinesia, freezing of gait. SPARK is for users to have their body motions monitored for dyskinesia and freezing gait while also having the user read paragraphs for speech analysis or pointing the camera for facial analysis. SPARK will also provide recommendations to physicians based on the collected information, such as changes in medication. SPARK is still being tested in order to determine the usefulness of such a framework for those who would use it. It is noted that there are limitations to it in that many patients will not be able to adopt the technologies required for it. There are also many issues, such as sensor misplacement and log errors, that can be encountered. Both these project show a desire to create health-based monitoring applications. 

%Current list of applications on smart watch, and their novelty. Current Prediction of BAC and associated ML algorithms