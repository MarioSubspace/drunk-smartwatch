\section{Introduction}
%%Describe the problem, state your contributions and the organization of the paper.T

Internet of Things (IoT) is a domain that represents the next most exciting technological revolution since the Internet. IoT will bring endless opportunities and impact every corner of our planet. In the healthcare domain, IoT promises to bring personalized health tracking and monitoring ever closer to the consumers. This phenomena is evidenced in a recent Wall Street Journal (June, 29, 2015) article entitled "Staying Connected is Crucial to Staying Healthy". Modern smartphones and related devices now contain more sensors than ever before. Data from sensors can be collected more easily and more accurately. In 2014, it is estimated that 46 million people are using IoT-based health and fitness applications. Currently, the predominant IoT-based health applications are in sports and fitness. However, disease management or preventive care health applications are becoming more prevalent. For example, the real-time preventive care applications such as those for detecting fall in elderly patients is one of the active research areas due to the aging population \cite{Tacconi:2011}. This paper describes a new category of real-time preventive health application that aims to reduce risky behaviors due to intoxication by predicting blood alcohol content (BAC) non-invasively and in real-time.

The Centers for Disease Control and Prevention (CDC) reported \cite{Schoenborn:2013} that between the years 2008 and 2010 roughly two-thirds of adults were current drinkers, but adults between the ages 18 and 24 were most commonly associated with consumption of more than 5 drinks in one day. There is also a known causal relationship between alcohol consumption and risky behaviors \cite{Assaad:2006}. One of the worst risky behaviors from drinking is drunk driving which is a persistent problem worldwide. Innocent people are harmed/killed in accidents where the driver was under the influence. It is reported by the Bureau of Transportation Statistics \cite{Chambers:2012} that in 2010 47.2\% of pedestrian fatalities and 39.9\% of vehicle occupant fatalities are caused by drunk driving.

Being able to calculate BAC is important for the portion of the population that consumes alcohol. Drinkers often find themselves in situations in which they are drinking but also need to plan on driving home. The issue comes when the drinker is not sure if they are sober enough, as in their BAC is below the legal limit, to drive home safely. In many cases, drinkers misjudge their intoxication, believing they will be able to safely get home, when actually their BAC is higher than the legal limit. While there are some ways available for people to measure their BAC, these methods are often invasive, inaccurate, or impractical. 

The most commonly known method of measuring BAC is the use of a breathalyzer. These devices are portable, easily accessible, and accurate. However, breathalyzers have a stigma of DUI arrest, fines, and license suspension. Breathalyzers are also non-discreet and invasive because the device requires the user to blow into the device in order to measure BAC. Due to these reasons, people often do not want to use such a device. A noninvasive, discreet, and real-time method of measuring BAC is desired.

In this paper, we propose an android-based IoT middleware architecture and application for collecting sensor data on smartwatches and smartphones to predict BAC in real-time. A study conducted in Washington University \cite{Ward:2015} shows that electronic intervention programs are more successful at reducing college student drinking than general alcohol awareness program. By wearing a smartwatch, a user can be alerted in real time when the BAC is above certain level and programmed with different options of intervention when such an event occurs. For example, the application can send a text message to a trusted family member or friend or can lock the ignition of the car engine of the drinker. Currently, there are no known applications that leverages data from common wearable devices to predict BAC non-invasively and in real time.

The main contributions of the paper are:
\begin{itemize}
	\item A middleware platform that can collect data from any type of device (virtual or physical) that can communicate with android phones. 
	\item A BAC prediction model based on live smartwatch sensor data that enables users to get BAC measurement in real time.
	\item Geographical visualization of BAC predictions.
	\item Collection of a labeled BAC data set for training the prediction model.
\end{itemize}

%****Summary of the results and conclusion****